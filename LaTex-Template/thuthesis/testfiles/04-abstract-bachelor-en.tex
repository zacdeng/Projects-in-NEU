\input{regression-test.tex}
\PassOptionsToClass{fontset=windows}{thuthesis}

\documentclass[degree=bachelor]{thuthesis}

\begin{document}
\START
\showoutput

\frontmatter
\setcounter{page}{2}

\begin{abstract*}
  Climate change has raised attention worldwide, whose impact on crop yield is closely concerning to food security.
  So it is vital to assess its impact by numerical simulation.

  Based on the calibration of ThuSPAC-Wheat and CERES-Wheat models, using the field experiment data of Yongledian Winter Wheat Station from 1999 to 2001, and the meteorological data from Beijing Weather Station from 1951 to 2006, the climate change impact on the potential wheat yield is studied.
  In addition, yields in different climate change scenarios are simulated.

  Model simulation using genetic parameters of Jingdong No. 8 shows that in ThuSPAC-Wheat Model, the wheat yield, the top weight and the LAI are well simulated, and in CERES-Wheat Model, the growth period and yield are well simulated.

  \thusetup{
    keywords* = {Climate change, yield, winter wheat, ThuSPAC-Wheat, CERES-Wheat},
  }
\end{abstract*}

\clearpage
\OMIT
\end{document}
